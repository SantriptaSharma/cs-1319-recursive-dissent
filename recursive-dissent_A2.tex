\documentclass{article}
\usepackage{graphicx} % Required for inserting images

\title{CS1319: PLDI - Assignment 1}
\author{Hrsh Venket \& Santripta Sharma}
\date{September 2023}

\begin{document}

\maketitle 


\section*{Explanation of Lexer}
Below is the code we have written for the lexer. This explanation therefore is our attempt to explain our choises by explaining how our code and the given rules for the lexer are equivalent.\\
\\
First let us consider the obvious cases, where we have written the regex for the lexer exactly as we have been given in the assignment.
\begin{verbatim}
KEYWORD char|else|for|if|int|return|void
PUNCT \[|\]|\(|\)|\{|\}|->|&|\*|\+|-|\/|%|!|\?|\<|\>|<=|>=|==|!=|&&|\\|\||=|:|;|,
\end{verbatim}
Here, we have simply written the regex as given in the assignment using a series of OR statements\\
\\
The explanation for string literals also directly follows from the definition given in the assignment.
\begin{verbatim}r
ESCAPE \\'|\\\?|\\\"|\\\\|\\a|\\b|\\f|\\n|\\r|\\t|\\v
STRCHAR [^\"\\\n]|{ESCAPE}
STRLIT \"{STRCHAR}*\"
\end{verbatim}\\
We define string characters as either the escape sequence or a member or the character set except for the double quote, the backslash and the newline character. We do this as we have done for the rest of the lexer, using \verb|^| to denote 'everything except'.\\
\\
String literals are defined as 1 or more string characters within double quotes.\\
\\
Two less obious cases are the single and multiline comments. For the single line comments, we have written the regex as follows:
\begin{verbatim}
COMMENTSINGLE \/\/([^\n])*\n
\end{verbatim}
\\
Here, \verb|\/\/| denotes the opening \verb|//| for a single line comment. Inside the comment, we allow any characters except for the newline operator. The \verb|*| means we can have 0 or more of the any type of character. It must thereby terminate with a newline character \verb|\n|.\\
\\
In the case of multline comments, our code is as below:
\begin{verbatim}
COMMENTMULTI \/\*([^\*]|\*[^\/])*\*\/
\end{verbatim}
\\
The multilinecomment starts with \*, followed by 0 or more of any character except for * or */, and is closed by */. This is to ensure that the comments start on the open multiline and end on the last close multiline.\\
\\
Now we can consider the less direct cases.
\begin{verbatim}
IDENT [a-zA-Z-][0-9a-zA-Z-]*
\end{verbatim}
By definition, the identifier cannot start with a digit. Therefore, we define that the identifier must start with an identifier non-digit, \verb|[a-zA-Z-]|. This is followed by 0 or more identifier characters, \verb|[0-9a-zA-Z-]|, which can be digits or non-digits
\begin{verbatim}
CHAR [^\\'\n]|{ESCAPE}
CONST ([\+-]?[1-9][0-9]*)|[0-9]+|'{CHAR}+'
\end{verbatim}
Here, we describe const, using some of the definitions given in the assignment. We define \verb|ESCAPE| as the escape sequence, which is relativel self explanatory. We define \verb|CHAR| as any character except for the singe quote and the newline character OR the escape sequence. We define \verb|CONST| as the constant, which can be a number, a character or a string.\\
\\
We therefore define const as 1 or more characters (or a character sequence), or an integer (defined with optional sign, starting with a non-zero digit and followed by 0 or more digits from 0-9). Finally, this also allows a random permutation of digits (0-9).\\
\\

\end{document}
